\chapter*{O olhar do errante sobre a cidade}
\addcontentsline{toc}{chapter}{O olhar do errante sobre a cidade, \emph{por Luis S. Krausz}}

\hfill\emph{Luis S. Krausz}

\bigskip

É em torno da figura de alguém que se entrega aos labirintos de
circulação urbana em busca de algo tão perdido quanto indefinível que se
constitui a narrativa de \emph{Fragmentos de um diário encontrado}. Obra
de 1932, ambientada em Paris, do escritor romeno Mihail Sebastián
(1907-1945), emoldurada por uma \emph{mise en abîme} -- um breve prólogo
em que o narrador afirma tratar-se de excertos de um ``caderno de capa
preta, lustrosa, de lona, igual àqueles que costumam ser usados, nas
mercearias, como livro-caixa'', que teria encontrado na ponte Mirabeau.

É através das passagens de um diário, cujo
autor anônimo encarna, de maneira exemplar, o olhar do errante sobre a
cidade e seus habitantes: percorre os bairros parisienses e, em desprezo
pelas convenções, busca uma grandeza e uma suposta genialidade perdidas.
Afirma desprezar a moralidade e estar em busca da santidade, isto é, de
uma epifania, no sentido \emph{nietzschiano} do termo, experimentando a
``vida nua'', metaforizada pela imagem do marinheiro que segue à deriva
pelo rio da existência.

O texto, assim, é de um dionisíaco para quem, ``entre um arbusto que
cresce selvagem e um jardineiro com tesouras e ideias, minha simpatia de
animal vai para o arbusto, por inteiro.'' ``Não quero prevenir coisa
nenhuma e não quero corrigir nada,'' escreve ele, igualmente. Ou: ``o
vento que sopra, eu lhe ofereço meu rosto. Que me penetre como se eu
fosse uma árvore na estepe, que me açoite à vontade e cesse quando bem
entender!''

Nada revolta este autor mais do que o pensamento cartesiano, o sonho da
razão que acaba por produzir monstros: ``odeio esse tal de Descartes,
pois ele (...) jamais teve, não, jamais pôde ter o calafrio de
pressentir a santidade. Era um jardineiro.'' A santidade de que se fala
aqui nada tem a ver com aquela proposta pela metafísica cristã ou pela
moralidade burguesa: é a concebida pelo shivaísmo e pelo dionisismo, que
busca a instauração do novo, numa espécie de sensualismo da
contracultura \emph{avant la letttre}, que almeja a intensidade e a
efemeridade da existência livre.

O destino do diário, apenas casualmente encontrado pelo narrador, que se
apresenta como uma espécie de ``catador de trapos'', representa,
implicitamente, o malogro dessa odisseia anônima e sua derrota ante o
caráter implacável da maquinaria do mundo. Mas é justamente em sua
renúncia a qualquer tipo de transcendência, de futuro, de heroísmo ou de
projeto, que se encontra o cerne das intenções do autor do diário: o
desmembramento e o despedaçamento são aspectos intrínsecos e
inseparáveis do dionisismo, assim como a aceitação da transitoriedade e
a aversão a tudo o que almeja à perenidade.

O texto torna-se, assim, um grito pela vida em si mesma, afinado com o
caráter rebelde das vanguardas artísticas europeias das décadas de 1920
e 1930, que influenciaram Sebastián tanto quanto outros literatos
romenos de seu tempo, tais como Cioran, Ionesco e Eliade, ao lado dos
quais ele participou do movimento estético Criterion. Este, porém, não
tardou a sofrer a influência da filosofia de Ionesco, uma mistura de
nacionalismo, existencialismo e misticismo cristão, assim como da Guarda
de Ferro, organização paramilitar fascista e ferrenhamente antissemita.
Como era judeu, Sebastián passou a ser excluído e execrado.

A publicação da obra de Sebastián traz de volta à atenção do público um
autor importante no cenário literário romeno, injustamente excluído da
posteridade tanto quanto o elusivo protagonista dessa narrativa.
