
\textbf{Mihail Sebastian} (1907--1945) foi dramaturgo, jornalista, ensaísta e romancista romeno. Sua escrita é afinada com o caráter rebelde das vanguardas artísticas europeias das décadas de 1920 e 1930, que influenciaram o autor tanto quanto outros literatos compatriotas de seu tempo, como Emil Cioran, Eugène Ionesco e Mircea Eliade. Não tardou a ser também influenciado pela filosofia de Nae Ionescu, mistura de nacionalismo, existencialismo e misticismo cristão, e da Guarda de Ferro, organização paramilitar fascista e antissemita. No entanto, por ser judeu, Sebastian não teve o reconhecimento de seus contemporâneos, e passou a ser excluído e execrado desse círculo. A publicação da obra de Sebastian traz de volta à atenção do público um autor importante no cenário literário romeno, injustamente excluído da posteridade tanto quanto o protagonista dessa narrativa.


\textbf{Fragmentos de um diário~encontrado}, narrativa escrita em 1932, é trazida a público através de um tradutor anônimo, que encontra na ponte Mirabeau, em Paris, ``um caderno de capa preta, lustrosa, de lona'', com ``leitura curiosa, passagens obscuras, anotações que pareceram estranhas ou absolutamente impróprias'': traduzido, ainda segundo a nota, de forma desastrada. Os relatos das aventuras do personagem"-autor, também anônimo, apresentam ao leitor uma figura que se entrega aos labirintos da cidade em busca de algo tão perdido quanto indefinível. E através das passagens de seu diário pessoal encarna o olhar do errante sobre a cidade e suas relações.


\textbf{Fernando Klabin} nasceu em São Paulo e formou"-se em Ciência Política pela Universidade de Bucareste, onde foi agraciado com a Ordem do Mérito Cultural da Romênia no grau de Oficial em 2016. Além de traduzir do romena, exerce atividades ocasionais como fotógrafo, escritor, ator e artista plástico.

\textbf{Luis S.\,Krausz} é professor de Literatura Hebraica e Judaica da \versal{USP}, ensaísta e tradutor. %Seus últimos livros publicados são: \emph{Entre exílio e redenção: aspectos da literatura de imigração judaico"-oriental} (2019) e \emph{Santuários heterodoxos: heresia e subjetividade na literatura judaica da Europa Central} (2017).

%\textbf{Gabriel Neistein Lowczyk} cursa graduação na Faculdade de Arquitetura e Urbanismo da \versal{USP}, desenvolve trabalhos e experimentações nos campos das artes visuais e na música.

